\section{KahaDB: A primer}
	\subsection{What is KahaDB?}
	KahaDB is an embedded lightweight non-relational database

	\subsection{How does it store messages?}
	Messages are stored in file-based data logs. When all of the messages in a data log have been successfully consumed, the data log is marked as deletable. At a predetermined clean-up interval, logs marked as deletable are removed from the system.


	\subsection{What are the files in KahaDB directory?}
		\begin{description}
			\item[db.data] \hfill \\
				The metadata store, mainly B-tree index giving message locations in the data logs. The file is periodically updated from the metadata cache (which is in memory).
			\item[db-n.log] \hfill \\
				These are the journal log files. This is where activeMQ messages will be stored.
			\item[db.free] \hfill \\
				Keeps track of free pages in db.data
			\item[db.redo] \hfill \\
				This is a redo log used to prevent partial page writes from making the file inconsistent.
		\end{description}


	\subsection{Recovering a corrupted metadata store}
	If the metadata store somehow becomes irretrievably corrupted, you can force recovery as follows (assuming the journal's data logs are clean):

	\begin{itemize}
    	\item While the broker is shut down, delete the metadata store, db.data.
    	\item Start the broker.
    	\item The broker now recovers by re-reading the entire journal and replaying all of the events in the journal in order to recreate the missing metadata.
    \end{itemize}
	While this is an effective means of recovering, you should bear in mind that it could take a considerable length of time if the journal is large.